\documentclass{article}

\usepackage{neurips_2019_author_response}
%

%%%%%%%%%%%%%%%%%%%%%%%%%%%%
% Paper dependent stuff    %
%%%%%%%%%%%%%%%%%%%%%%%%%%%%

\newcommand{\FTQ}{\textcolor{myalgocolor}{\normalfont \texttt{FTQ}}\xspace}
\newcommand{\FTQl}{\textcolor{myalgocolor}{\normalfont \texttt{FTQ}$(\lambda)$}\xspace}
\newcommand{\BFTQ}{\textcolor{myalgocolor}{\normalfont \texttt{BFTQ}}\xspace}
\newcommand{\ov}{\overline}
\newcommand{\oa}{\ov{a}}
\newcommand{\ox}{\ov{x}}
\newcommand{\oz}{\ov{z}}
\newcommand{\oy}{\ov{y}}
\newcommand{\os}{\ov{s}}
\newcommand{\ocS}{\ov{\cS}}
\newcommand{\ocA}{\ov{\cA}}

%%%%%%%%%%%%%%%%%%%%%%%%%%%%
% Aesthetics               %
% over-underline, hat, bold%
%%%%%%%%%%%%%%%%%%%%%%%%%%%%

\newcommand{\eps}{\varepsilon}
\newcommand{\vareps}{\varepsilon}
\renewcommand{\epsilon}{\varepsilon}
%\renewcommand{\hat}{\widehat}
\renewcommand{\tilde}{\widetilde}
\renewcommand{\bar}{\overline}

\newcommand*{\MyDef}{\mathrm{\tiny def}}
\newcommand*{\eqdefU}{\ensuremath{\mathop{\overset{\MyDef}{=}}}}% Unscaled version
\newcommand*{\eqdef}{\mathop{\overset{\MyDef}{\resizebox{\widthof{\eqdefU}}{\heightof{=}}{=}}}}


\def\:#1{\protect \ifmmode {\mathbf{#1}} \else {\textbf{#1}} \fi}
\newcommand{\CommaBin}{\mathbin{\raisebox{0.5ex}{,}}}

\newcommand{\wt}[1]{\widetilde{#1}}
\newcommand{\wh}[1]{\widehat{#1}}
\newcommand{\wo}[1]{\overline{#1}}
\newcommand{\wb}[1]{\overline{#1}}

% bf and bm missing due to conflict!!
\newcommand{\bsym}[1]{\mathbf{#1}}
\newcommand{\bzero}{\mathbf{0}}
\newcommand{\ba}{\mathbf{a}}
\newcommand{\bb}{\mathbf{b}}
\newcommand{\bc}{\mathbf{c}}
\newcommand{\bd}{\mathbf{d}}
\newcommand{\be}{\mathbf{e}}
\newcommand{\bbE}{\mathbb{E}}
\newcommand{\bg}{\mathbf{g}}
\newcommand{\bh}{\mathbf{h}}
\newcommand{\bi}{\mathbf{i}}
\newcommand{\bj}{\mathbf{j}}
\newcommand{\bk}{\mathbf{k}}
\newcommand{\bl}{\mathbf{l}}
\newcommand{\bn}{\mathbf{n}}
\newcommand{\bo}{\mathbf{o}}
\newcommand{\bp}{\mathbf{p}}
\newcommand{\bq}{\mathbf{q}}
\newcommand{\br}{\mathbf{r}}
\newcommand{\bs}{\mathbf{s}}
\newcommand{\bt}{\mathbf{t}}
\newcommand{\bu}{\mathbf{u}}
\newcommand{\bv}{\mathbf{v}}
\newcommand{\bw}{\mathbf{w}}
\newcommand{\bx}{\mathbf{x}}
\newcommand{\by}{\mathbf{y}}
\newcommand{\bz}{\mathbf{z}}

\newcommand{\bA}{\mathbf{A}}
\newcommand{\bB}{\mathbf{B}}
\newcommand{\bC}{\mathbf{C}}
\newcommand{\bD}{\mathbf{D}}
\newcommand{\bE}{\mathbf{E}}
\newcommand{\bF}{\mathbf{F}}
\newcommand{\bG}{\mathbf{G}}
\newcommand{\bH}{\mathbf{H}}
\newcommand{\bI}{\mathbf{I}}
\newcommand{\bJ}{\mathbf{J}}
\newcommand{\bK}{\mathbf{K}}
\newcommand{\bL}{\mathbf{L}}
\newcommand{\bM}{\mathbf{M}}
\newcommand{\bN}{\mathbf{N}}
\newcommand{\bO}{\mathbf{O}}
\newcommand{\bP}{\mathbf{P}}
\newcommand{\bQ}{\mathbf{Q}}
\newcommand{\bR}{\mathbf{R}}
\newcommand{\bS}{\mathbf{S}}
\newcommand{\bT}{\mathbf{T}}
\newcommand{\bU}{\mathbf{U}}
\newcommand{\bV}{\mathbf{V}}
\newcommand{\bW}{\mathbf{W}}
\newcommand{\bX}{\mathbf{X}}
\newcommand{\bY}{\mathbf{Y}}
\newcommand{\bZ}{\mathbf{Z}}

% calligraphic
\newcommand{\cf}{\mathcal{f}}
\newcommand{\cA}{\mathcal{A}}
\newcommand{\cB}{\mathcal{B}}
\newcommand{\cC}{\mathcal{C}}
\newcommand{\cD}{\mathcal{D}}
\newcommand{\cE}{\mathcal{E}}
\newcommand{\cF}{\mathcal{F}}
\newcommand{\cG}{\mathcal{G}}
\newcommand{\cH}{\mathcal{H}}
\newcommand{\cI}{\mathcal{I}}
\newcommand{\cJ}{\mathcal{J}}
\newcommand{\cK}{\mathcal{K}}
\newcommand{\cL}{\mathcal{L}}
\newcommand{\cM}{\mathcal{M}}
\newcommand{\cN}{\mathcal{N}}
\newcommand{\cO}{\mathcal{O}}
\newcommand{\cP}{\mathcal{P}}
\newcommand{\cQ}{\mathcal{Q}}
\newcommand{\cR}{\mathcal{R}}
\newcommand{\cS}{\mathcal{S}}
\newcommand{\cT}{\mathcal{T}}
\newcommand{\cU}{\mathcal{U}}
\newcommand{\cV}{\mathcal{V}}
\newcommand{\cW}{\mathcal{W}}
\newcommand{\cX}{\mathcal{X}}
\newcommand{\cY}{\mathcal{Y}}
\newcommand{\cZ}{\mathcal{Z}}

%%%%%%%%%%%%%%%%%%%%%%%%%%%%
% Math jargon              %
%%%%%%%%%%%%%%%%%%%%%%%%%%%%
\newcommand{\wrt}{w.r.t.\xspace}
\newcommand{\defeq}{\stackrel{\mathclap{\normalfont\mbox{\tiny def}}}{=}}
\newcommand{\maxund}[1]{\max\limits_{#1}}
\newcommand{\supund}[1]{\text{sup}\limits_{#1}}
\newcommand{\minund}[1]{\min\limits_{#1}}
\renewcommand{\epsilon}{\varepsilon}
\newcommand{\bigotime}{\mathcal{O}}


\DeclareMathOperator*{\argmin}{arg\,min}
\DeclareMathOperator*{\argmax}{arg\,max}
\DeclareMathOperator*{\cupdot}{\mathbin{\mathaccent\cdot\cup}}

%%%%%%%%%%%%%%%%%%%%%%%%%%%%
% Matrix operators         %
%%%%%%%%%%%%%%%%%%%%%%%%%%%%
\newcommand{\transpose}{^\mathsf{\scriptscriptstyle T}}
\newcommand{\transp}{\mathsf{\scriptscriptstyle T}}

%%%%%%%%%%%%%%%%%%%%%%%%%%%%
% Statistic operators      %
%%%%%%%%%%%%%%%%%%%%%%%%%%%%
\newcommand{\probability}[1]{\mathbb{P}\left(#1\right)}
\newcommand{\probdist}{Pr}
\DeclareMathOperator*{\expectedvalue}{\mathbb{E}}
\DeclareMathOperator*{\variance}{\text{Var}}
\newcommand{\expectedvalueover}[1]{\expectedvalue\limits_{#1}}
\newcommand{\condbar}{\;\middle|\;}
\newcommand{\gaussdistr}{\mathcal{N}}
\newcommand{\uniformdistr}{\mathcal{U}}
\newcommand{\bernoullidist}{\mathcal{B}}

%%%%%%%%%%%%%%%%%%%%%%%%%%%%
% Algebraic Sets           %
%%%%%%%%%%%%%%%%%%%%%%%%%%%%
\newcommand{\Real}{\mathbb{R}}
\newcommand{\Natural}{\mathbb{N}}
\newcommand{\statespace}{\mathcal{X}}
\newcommand{\funcspace}{\mathcal{F}}
\newcommand{\dynaspace}{\mathcal{T}}

%
%\newtheorem{theorem}{Theorem}
%\newtheorem{definition}{Definition}
%\newtheorem{lemma}{Lemma}
\newtheorem{proposition}{Proposition}
\newtheorem{remark}{Remark}
\newtheorem{conjecture}{Conjecture}
%\newtheorem{property}{Property}
%\newtheorem{assumption}{Assumption}
%\newtheorem{conjecture}{Conjecture}
%
%\newtheorem*{definition*}{Definition}
%\newtheorem*{theorem*}{Theorem}
%\newtheorem*{proposition*}{Proposition}
%\newtheorem*{remark*}{Remark}
\usepackage[utf8]{inputenc} % allow utf-8 input
\usepackage[T1]{fontenc}    % use 8-bit T1 fonts
\usepackage{hyperref}       % hyperlinks
\usepackage{url}            % simple URL typesetting
\usepackage{booktabs}       % professional-quality tables
\usepackage{amsfonts}       % blackboard math symbols
\usepackage{nicefrac}       % compact symbols for 1/2, etc.
\usepackage{microtype}      % microtypography
\usepackage{todonotes}
\long\def\td#1{\todo[inline]{#1}}

\usepackage{lipsum}

\begin{document}

%\section{Common}

First of all, we would like to thank the reviewers for their valuable comments, which will help us improve the revised manuscript. We endeavour to address all the major remarks below.

Second, we would like to notify all the reviewers about a major theoretical improvement allowing us to transform the conjecture at line 107 into a theorem. At submission time, the proof was missing two steps (lines 443--444 and 465), and we decided to present it as a conjecture for the sake of caution. In the meantime, we have proved them and we propose to make it a theorem.
%The \textbf{main concern} expressed was perhaps that of Reviewer \#3 regarding the presentation of the result on the optimality operator. The reason for this is that the proof was still incomplete at the time of the submission deadline. Two small parts were missing: l443-444 and l465. Hence, by way of precaution, we preferred to present this property as a conjecture with a sketch of proof rather than a result. Our intention was to convey to the reader an intuition of why we do observe empirical convergence despite our negative result regarding contraction in the general case, though details were missing. The proof has been completed since, and the contractivity result on smooth functions is now presented as a theorem.

Third, a remark common to all reviewers is that Algorithm 3 (convex hull policy) lacks a detailed explanation, which we acknowledge: it was a questionable decision due to space constraints. We will include a descriptive paragraph.


\paragraph{\large Reviewer \#1}

%We will add the missing appendices.

%The framework searches for a minimum cost policy given a maximum reward policy satisfying budget restrictions. In principle it could be framed the other way around searching for a maximum reward policy given minimum cost. Add a note of why this is not a good idea.

\textbf{In principle it could be framed [...] why this is not a good idea.}
Swapping the min and max would lead to a different set of policies. Indeed,  the BMDP solution is generally not a saddle point: max-min $<$ min-max. % and make the constraint pointless (a minimum-cost policy satisfies any feasible budget). The true objective in a BMDP is really to maximise the rewards under cost constraint, and the cost minimisation is rather an artefact only introduced to break the ties in the definition of $V_c^*$.


\textbf{When creating batch samples what happens when the budget is zero? The episodes ends with a negative reward?} If the agent explores/exploits with zero budget, it will always select the estimated safest action, with minimal estimated expected cost $Q_c$. If costs are still incurred, the model $Q_\theta$ will be updated accordingly.

\textbf{As it is not a contraction in general, in which cases
should we not use the proposed approach?} The Remark 1 and the counter-example in Theorem 2 both indicate that the proposed approach is more likely to diverge when $Q^*$ is steep, i.e. in problems where a small increase in budget $\beta$ leads to substantial gains in rewards. As in most theoretical works, the required assumptions may be transgressed in real world. But, it does not necessarily imply that the algorithm would not work in practice. For a sensitive real world application, we would not refrain from using the algorithm, but would strongly advise to supervise it.% For a sensitive real world application, we would not refrain from using the algorithm, but recommend setting strong safety margins on the budget $\beta$.

\textbf{How can the budget lie between two $Q_c$, is it not the case that all of
them should respect the budget constraint?}
$Q^*_c(\overline{s},\overline{a})$ is the cost induced by \textit{first} executing action $\overline{a}$ and only \textit{then} following the optimal budgeted policy $\pi^*$. Among all these actions, some of them may not be feasible and exceed the budget $\beta$, which is why an additional optimisation step is required in the definition of the greedy policy.
%TU: To dodge the "expectation weakness"
%can we tune the mixture to obtain safer policies (how "safer" being left as future work) ?

\paragraph{\large Reviewer \#2}

%\td{I am confused as to why/how the $\beta$ evolves during the learning process and how this effects the goals one has when using a BMDP. Do we not care about how much risk is induced (or negative effects experienced) during training? Moreover, if the policy selects a new $\beta$ at each new time step, how do we enforce a *specific* $\beta$ when we wish to *use* the policy?}
\textbf{(A.1)} 
The cost constraints only apply to the final trained budgeted policy, we indeed do not care about the costs incurred during training. To enforce a specific $\beta$ at test time, set the initial augmented state to $\overline{s}_0 =(s_0,\beta)$.

%\td{The convex hull procedure seems critical to being able to actually use the algorithm, but the explanation is lacking any intuitive interpretation. For example, it's not clear to me from 4.1 and Algorithm 3 exactly how "Budget $\beta$ [is] always respected" (comment on Algorithm 3, line 9). Can the authors provide more explanation/intuition about what is going on here?}

\textbf{(A.2)} If we reach Line 9 in Algorithm 3, it must be that the condition $\beta < q_c^2$ in Line 5 never holds, which means that every action verifies $q_c \leq \beta$: it is a case where the budget is always respected.

% (3) The authors state that "The pseudo-code of our exploration procedure is shown in Algorithm 4 in Appendix B." Since this component of your algorithm is one of the main hypotheses that is being validated, it should appear in the main paper. Suggest swapping for Algorithms 1/2 since these are basic extensions of existing techniques.

\textbf{(A.3)} We will follow your suggestion: indeed Algorithm 4 is more specific to our work than Algorithms 1/2.

%\td{ There is a mismatch between the introduction of the paper, which speaks generally to using BMDPs, and the actual experiments run, which show the benefit of using BMDPs *compared to* computing a set of solutions using existing techniques like FTQ($\lambda$). The introduction needs to mention that approaches like the latter *are* available solutions and frame the contribution of the paper rather as one of providing a "better" solution in whichever way the authors feel this is best described (more-efficient, etc.).}
\textbf{(B)} We will make the introduction clearer and properly introduce the "set of solutions" techniques like FTQ($\lambda$) along with a proper comparison to the BMDP approach. Namely, they do not recover optimal budgeted policies, in addition to being inefficient.   
%\textbf{(B)} FTQ($\lambda$) is a good heuristic but is not optimal for BMDPs (in addition to being inefficient). To be convinced of this: an MDP (obtained after relaxation) always admits a deterministic solution, whereas it is not necessarily the case for a CMDP and hence a BMDP.

%\td{Several times in the paper, it is mentioned that experiments are done in "two environments," but aren't there three?}

\textbf{Minor comment 2} Indeed, but one of them is just a toy example to illustrate the risk-sensitive exploration only.

%\td{The definition in (2) is odd given that you say the "budget evolves as part of the dynamics." That is, if $\beta'=\beta_a$ (as suggested by the Dirac function, then it is merely whatever the action says it was, correct? Why is that part of "the dynamics?"}

\textbf{Minor comment 3} This is absolutely correct. The problem is that since we framed the problem in an augmented space where the budget $\beta$ is part of the state $\overline{s}$, its evolution can only be described within the dynamics of $\overline{s}$ (which we specify). We agree that this makes the notations quite obscure for a very simple idea, but on the other hand casting the problem in such a way drastically simplifies the next definitions and proofs, e.g. that of Proposition 1.

\paragraph{\large Reviewer \#3}

\textbf{Presentation of the result on the optimality operator.} The main criticism concerned the presentation of the contractivity of $\mathcal{T}$ over smooth Q-functions as a conjecture rather than a proven result. We hope that these reservations have been lifted by the upgrade of the conjecture to a theorem.

\textbf{The exploration strategy from Section 3.2 is merely described}: This strategy is motivated by the observation that a wide variety of risk levels needs to be experienced during training, which can be achieved by enforcing the risk constraints during data collection. This intuition was meant to be conveyed by the corridor example where a conventional greedy exploration procedure fails to visit the safe region. We will add an additional discussion to clarify this point.

\textbf{The implementation proposed in Section 4 seems to rely heavily on results from Boutilier and Lu, 2016, but a lot of the necessary context is missing}. Several differences exist between the approach from Boutilier and Lu (2016)  - the greedy budget allocation (GBA) algorithm - and ours: they rely on a finite set of non-dominated budget points $B$ which grows exponentially with the horizon and becomes uncountably infinite in continuous state spaces. They also compute and sort a matrix of bang-per-buck ratios of size $|S|\times B$, which is again infeasible when $S$ is continuous. In contrast, we instead rely on estimating the optimal cost-to-go $Q_c^*$, which requires an additional min constraint in its definition that does not appear in Boutilier and Lu (2016) (they only estimate $V_r^*$).

\end{document}

