\documentclass[slideopt,A4,showboxes,svgnames]{beamer}

%% list of packages here
\usepackage[absolute,showboxes,overlay]{textpos}
\usepackage{mathtools}
\usepackage{amsmath}
\usepackage{amssymb}
\usepackage{pifont}
\usepackage{multicol}
\usepackage[linesnumbered,ruled,vlined]{algorithm2e}

\usepackage{theme/beamerthemeinria}

\input{mathdef}


\input{theme/style}

\title[Budgeted Reinforcement Learning]{Budgeted Reinforcement Learning\\in Continuous State Space}
\subtitle{Sous-titre}
\date[date]{date}
\author[Carrara et al.]{Nicolas Carrara\inst{1},
	Edouard Leurent\inst{1,2},\\
	Tanguy Urvoy\inst{3},
	Romain Laroche\inst{4},\\
	Odalric Maillard\inst{1},
	Olivier Pietquin\inst{1,5}}
\institute{
	\inst{1} Inria SequeL, 
	\inst{2} Renault Group,\\
	\inst{3} Orange Labs,
	\inst{4} Microsoft Montr\'eal,\\
	\inst{5} Google Research, Brain Team}

\begin{document}

\begin{frame}
    \titlepage
\end{frame}


\frame{\tocpage}
 
\section{Setting}
\frame{\sectionpage}


\begin{frame}{Motivation}

\begin{alertblock}{Conflicting Objectives}
Single scalar reward to represent multiple contradictory aspects (task completion vs safety)

No explicit control over the tradeoffs
\end{alertblock}

Examples?
\end{frame}

\begin{frame}{Examples}
\begin{alertblock}{Dialogue systems}
A slot-filling problem: the agent fills a form by asking the user each slot. It can either:
\begin{itemize}
	\item ask to answer using {\green voice} {\green (safe/slow)};
	\item ask to answer with a {\red numeric pad} {\red (unsafe/fast)}.
\end{itemize}
\end{alertblock}

\begin{alertblock}{Autonomous Driving}
The agent is driving on a two-way road with a car in front of it,
\begin{itemize}
\item it can {\green stay behind} {\green (safe/slow)};
\item it can {\red overtake} {\red (unsafe/fast)}.
\end{itemize}
\begin{center}
\href{https://budgeted-rl.github.io/assets/highway-neutral.gif}{\includegraphics[width=0.5\linewidth]{img/highway_env}}
\end{center}
\end{alertblock}
\end{frame}


\begin{frame}{Constrained Reinforcement Learning}
\begin{alertblock}{\only<2>{Constrained} Markov Decision Process}
	A\only<1>{n} \alert<2>{\only<2>{C}MDP} is a tuple $(\cS, \cA, P, \Rr, \only<2>{\alert<2>{R_c,}} \gamma \only<2>{\alert{, \beta}})$ with:
	\begin{multicols}{2}
			\begin{itemize}
			\item Rewards $\Rr\in\Real^{\cS \times \cA}$
			\item[]
			\item<2> \alert<2>{Costs $R_c\in\Real^{\cS \times \cA}$}
			\item<2> \alert{Budget $\beta$}
		\end{itemize}
	\end{multicols}
\end{alertblock}
\begin{block}{Objective}
Maximise rewards \only<2>{\alert{while keeping costs under a fixed budget}}
	\begin{equation*}
	\begin{array}{rl}
	 \max_{\pi\in\cM(\cA)^{\cS}} &\expectedvalue\left[\sum_{t=0}^\infty \gamma^t \Rr(s_t, a_t) \condbar s_0=s \right]\\
	\visible<2>{
	\text{ s.t. }  \displaystyle &\alert<2>{\expectedvalue\left[\sum_{t=0}^\infty \gamma^t R_c(s_t, a_t) \condbar s_0=s\right] \leq \beta}}
	\end{array}
	\end{equation*}
\end{block}
\end{frame}

\begin{frame}{Budgeted Reinforcement Learning}
\begin{alertblock}{Budgeted Markov Decision Process}
	A \alert{BMDP} is a tuple $(\cS, \cA, P, \Rr, \Rc, \gamma\alert{, \cB})$ with:
	\begin{multicols}{2}
		\begin{itemize}
			\item Rewards $\Rr\in\Real^{\cS \times \cA}$
			\item[]
			\item Costs $\Rc\in\Real^{\cS \times \cA}$
			\item \alert{Budget space $\cB$}
		\end{itemize}
	\end{multicols}
\end{alertblock}
\begin{block}{Objective}
Maximise rewards while keeping costs under an \alert{adjustable} budget. $\alert{\forall \beta\in\cB},$
	\begin{align*}
	\begin{array}{rl}
	 \max_{\pi\in\cM(\cA \alert{\times \cB})^{\cS\alert{\times \cB}}} &\expectedvalue\left[\sum_{t=0}^\infty \gamma^t \Rr(s_t, a_t) \condbar s_0=s, \alert{\beta_0 = \beta}\right]\\
		\text{ s.t. } &\expectedvalue\left[\sum_{t=0}^\infty \gamma^t \Rc(s_t, a_t) \condbar s_0=s, \alert{\beta_0 = \beta}\right] \leq \beta
	\end{array}
	\end{align*}
\end{block}
\end{frame}

\begin{frame}{Augmented Settings}
Budgeted policies $\pi$
\begin{itemize}
	\item Take a budget $\beta$ as an additional input
	\item Output a next budget $\beta'$ 
	\item $\pi:\underbrace{(s,\beta)}_{\os} \rightarrow \underbrace{(a,\beta')}_{\oa}$
\end{itemize}

\end{frame}


\begin{frame}{Augmented Setting}

\begin{alertblock}{Definition (Augmented spaces)}
	\begin{itemize}
		\item States $\ocS = \cS\times\cB$.
		\item Actions $\ocA = \cA\times\cB$.
		\item Dynamics $\ov{P}$
		\vspace*{-0.5cm}
		\[\text{state } (s,\beta), \text{ action }(a, \beta_a) \rightarrow \text{ next state }\begin{cases}s'\sim P(s'|s, a)\\\beta' = \beta_a\end{cases}\]
	\end{itemize}
\end{alertblock}

\begin{alertblock}{Definition (Augmented signals)}
	\begin{enumerate}
		\item Rewards $R = ({\green R_r}, {\red R_c})$
		\item Returns $G^\pi = ({\green G_r^\pi}, {\red G_c^\pi}) \eqdef \sum_{t=0}^\infty \gamma^t R(\os_t,\oa_t)$
		\item Value $V^\pi(\os) = ({\green V_r^\pi}, {\red V_c^\pi}) \eqdef \expectedvalue\left[ G^\pi \condbar \ov{s_0} = \os\right]$
		\item Q-Value $Q^\pi(\os, \oa)= ({\green Q_r^\pi}, {\red Q_c^\pi}) \eqdef \expectedvalue\left[ G^\pi \condbar \ov{s_0} = \os, \ov{a_0} = \oa\right]$
	\end{enumerate}
\end{alertblock}
\end{frame}

\section{Budgeted Dynamic Programming}
\frame{\sectionpage}


\begin{frame}{Policy Evaluation}
\begin{proposition}[Budgeted Bellman Expectation]
	The Bellman Expectation equations are \alert{preserved}
	\begin{align*}
	V^\pi(\os) = \sum_{\oa\in\ocA}\pi(\oa | \os) Q^\pi(\os, \oa) \\ Q^\pi(\os, \oa) = R(\os, \oa) + \gamma\sum_{\os'\in\ocS}\ov{P}\left(\os' \condbar \os, \oa\right) V^\pi(\os') 
	\end{align*}
\end{proposition}
\pause
\begin{proposition}[Contraction]
	The Bellman Expectation Operator $\cT^\pi$ is a $\gamma$-contraction.
	\begin{align*}
	\cT^\pi Q(\os, \oa) &\eqdef R(\os, \oa) + \gamma \sum_{\os'\in\ocS}\sum_{\oa'\in\ocA}\ov{P}(\os'|\os, \oa)\pi(\oa'|\os') Q(\os', \oa')
	\end{align*}
\end{proposition}

\begin{itemize}
	\item[\green \checkmark] We can {\green evaluate} a budgeted policy $\pi$
\end{itemize}
\end{frame}

\begin{frame}{Budgeted Optimality}
\begin{alertblock}{Definition (Budgeted Optimality)}
	In that order, we want to:
	\begin{enumerate}
		\item[(i)] {\red Respect the budget $\beta$}: 
		\begin{equation*}
		\Pi_a(\os) \eqdef \{\pi\in\Pi: V_c^\pi(s, \beta) \hler{\leq \beta}\}
		\end{equation*}
		\item[(ii)] {\green Maximise the rewards}:
		\begin{equation*}
		V_r^*(\os) \eqdef \hleg{\max}_{\pi\in\Pi_a(\os)}  V_r^\pi(\os) \quad\quad \Pi_r(\os) \eqdef \hleg{\argmax}_{\pi\in\Pi_a(\os)}  V_r^\pi(\os)
		\end{equation*}
		\item[(iii)] \alert{Minimise the costs}: 
		\begin{equation*}
		V_c^*(\os) \eqdef \hleb{\min}_{\pi\in\Pi_r(\os)}  V_c^\pi(\os), \quad\quad \Pi^*(\os) \eqdef \hleb{\argmin}_{\pi\in\Pi_r(\os)}  V_c^\pi(\os)
		\end{equation*}
	\end{enumerate}
\end{alertblock}
We define the budgeted action-value function $Q^*$ similarly
\end{frame}

\begin{frame}{Budgeted Optimality}
\begin{theorem}[Budgeted Bellman Optimality Equation]
	$Q^*$ verifies the following equation:
	\begin{align*}
	Q^{*}(\os, \oa) &= \cT Q^{*}(\os, \oa)\\
	&\eqdef R(\os, \oa) + \gamma \sum_{\os'\in\ocS} \ov{P}(\ov{s'} | \os, \oa)\sum_{\ov{a'}\in \ocA} \pi_\text{greedy}(\ov{a'}|\ov{s'}; Q^*) Q^{*}(\ov{s'}, \ov{a'})
	\end{align*}
\end{theorem}
where the greedy policy {$\pi_\text{greedy}$} is defined by:
\begin{align*}
\pi_\text{greedy}(\oa|\os; Q) \in &\hleb{\argmin}_{\rho\in\Pi_r^Q} \expectedvalueover{\oa\sim\rho}Q_c(\os, \oa), \\
\text{where }\quad \Pi_r^Q \eqdef &\hleg{\argmax}_{\rho\in\cM(\ocA)} \expectedvalueover{\oa\sim\rho} Q_r(\os, \oa) \\
& \text{ s.t. }  \expectedvalueover{\oa\sim\rho} Q_c(\os, \oa) \hler{\leq \beta}
\end{align*}
\end{frame}

\begin{frame}{The optimal policy}
\begin{proposition}[Optimality of the policy]
	$\pi_\text{greedy}(\cdot~; Q^*)$ is \alert{simultaneously optimal} in all states $\os\in\ocS$: $$\pi_\text{greedy}(\cdot~; Q^*)\in \alert{\Pi^*(\os)}$$
	
	In particular, $V^{\pi_\text{greedy}(\cdot; Q^*)} = V^*$ and $Q^{\pi_\text{greedy}(\cdot; Q^*)}= Q^*$.
\end{proposition}

\begin{proposition}[Solving the non-linear program]
	$\pi_\text{greedy}$ can be computed \alert{efficiently}, as a mixture $\pi_\text{hull}$ of two points that lie on the convex hull of $Q$.
	\[\pi_{\text{greedy}}=\pi_\text{hull}\]
\end{proposition}
\end{frame}

\foreach \n in {0,1,2,3,4}{
	\begin{frame}{Solving the non-linear program: intuition}
	\begin{figure}
		\centering
		\includegraphics[scale=1.0,page=1]{img/b\n}
	\end{figure}
\end{frame}
}

\begin{frame}{Convergence analysis}
Recall what we've shown so far:
\[\cT \xrightarrow[fixed-point]{} Q^* \xrightarrow[tractable]{} \pi_{\text{hull}}(Q^*) \xrightarrow[equal]{} \pi_{\text{greedy}}(Q^*) \xrightarrow[optimal]{} \]

\pause
We're {\green almost there}! 

All that is left is to perform \alert{Fixed-Point Iteration} to compute $Q^*$.

\pause
\begin{theorem}[Non-Contractivity]
For any BMDP ($\cS,\cA,P,R_r,R_c,\gamma$) with $|\cA| \geq 2$, $\cT$ is \textbf{\red not} a contraction.
$$\forall\epsilon>0, \exists Q^1,Q^2\in(\Real^2)^{\ocS\ocA}:\|\cT Q^1-\cT Q^2\|_\infty \geq \frac{1}{\epsilon}\|Q^1-Q^2\|_\infty$$
\end{theorem}

\begin{itemize}
	\item[\red \xmark] We {\red cannot guarantee} the convergence of $\cT^n(Q_0)$ to $Q^*$
\end{itemize}
\end{frame}

\foreach \n in {0,1,2}{
	\begin{frame}{Not a contraction: intuition}
	\begin{figure}
		\centering
		\includegraphics[scale=0.9,page=1]{img/d\n}
	\end{figure}
	\end{frame}
}

\begin{frame}{Convergence analysis}
Thankfully,
\begin{theorem}[Contractivity on smooth Q-functions]
	$\cT$ \alert{is a contraction} when restricted to the subset $\cL_\gamma$ of $Q$-functions such that "$Q_r$ is $L$-Lipschitz with respect to $Q_c$", with $L<\frac{1}{\gamma}-1$.
\end{theorem}
\begin{equation*}
\cL_\gamma = \left\{\begin{array}{cc}
Q\in(\Real^2)^{\ocS\ocA}\text{ s.t. }\exists L<\frac{1}{\gamma}-1: \forall \os\in\ocS,\oa_1,\oa_2\in\ocA,   \\
|Q_r(\os,\oa_1) - Q_r(\os,\oa_2)| \leq L|Q_c(\os,\oa_1) - Q_c(\os,\oa_2)|
\end{array}\right\}
\end{equation*}
\begin{itemize}
	\item[\green \checkmark] We {\green guarantee} convergence under some (strong) assumptions
	\item[\green \checkmark] We {\green observe} empirical convergence
\end{itemize}
\end{frame}

\begin{frame}{Budgeted Dynamic Programming}
\begin{center}
	\scalebox{0.9}{
		\begin{algorithm}[H]
			\DontPrintSemicolon
			\KwData{$P, R_r, R_c$}
			\KwResult{$Q^*$}
			$Q_{0} \leftarrow 0$\;
			\Repeat{convergence}{
				$Q_{k+1} \leftarrow \cT Q_k$\;
			}
			\caption{Budgeted Value-Iteration}
		\end{algorithm}
	}
\end{center}
\end{frame}

\section{Budgeted Reinforcement Learning}
\frame{\sectionpage}

\begin{frame}{Extension to the RL setting}

We address several limitations of Budgeted Value-Iteration

\begin{enumerate}[<+->]
	\item If the $P$, $R_r$ and $R_c$ are unknown:
	\begin{itemize}[<+->]
		\item Work with a {batch} of samples $\cD=\{(\os_i,\oa_i,r_i,\os_i'\}_{i\in [0,N]}$
		\item Replace $\cT$ with a sampling operator $\hat{\cT}$:
		\begin{equation*}
		\hat{\cT} Q(\os_i, \oa_i, r_i, \os'_i) \eqdef r_i + \gamma \sum_{\ov{a'_i}\in \cA_i} \pi_\text{greedy}(\ov{a'_i}|\ov{s'_i}; Q) Q(\ov{s'_i}, \ov{a'_i}).
		\end{equation*}
	\end{itemize}
	\item If $\cS$ is continuous:
	\begin{itemize}
		\item Employ function approximation $Q_\theta$, and minimise a regression loss
		$$\cL(Q_\theta, Q_\text{target};\cD) = \sum_{\cD} ||Q_\theta(\os, \oa) - Q_\text{target}(\os, \oa, r, \os')||_2^2$$
	\end{itemize}
	
\end{enumerate}

\end{frame}

\begin{frame}{More scaling}

\begin{itemize}[<+->]
\item CPU parallel computing of the targets $\sum_{\ov{a'_i}\in \cA_i} \pi_\text{greedy}(\ov{a'_i}|\ov{s'_i}; Q) Q(\ov{s'_i}, \ov{a'_i})$
\item Same for interactions with the environment.
\item Neural Network for function approximation: 
\begin{center}
	\resizebox{.5\textwidth}{!}{%
		\input{../../source/architecture.tex}
	}
\end{center}
\end{itemize}
\end{frame}

\section{Experiments}
\frame{\sectionpage}

\begin{frame}{A baseline approximate solution}
\begin{alertblock}{Lagrangian Relaxation}
	Consider the dual problem so as to replace the hard constraint by a soft constraint penalised by a \alert{Lagrangian  multiplier $\lambda$}:
	\begin{equation*}
	\label{eq:lagrangian-relaxation}
	\max_\pi \expectedvalue \sum_t {\green \gamma^t R_r(s,a)} - \alert{\lambda}{\red \gamma^t R_c(s,a)}
	\end{equation*}
	\begin{itemize}
		\item Train many policies $\pi_k$ with penalties $\lambda_k$ and recover the cost budgets $\beta_k$% (Theorem 4.4, \cite{Beutler85})
		\item Very data/memory-heavy
	\end{itemize}
\end{alertblock}
\end{frame}


\begin{frame}{Dialogue systems}
A slot-filling problem: the agent (the dialogue system) fills a form by asking the user each slot. It can either:
\begin{itemize}
	\item ask to answer using {\green voice} {\green (safe/slow)};
	\item ask to answer with a {\red numeric pad} {\red (unsafe/fast)}.
\end{itemize}


\begin{center}
	\includegraphics[trim={0 0.1cm 0 0.8cm}, clip, width=0.6\textwidth]{../../source/img/slot-filling.pdf}
\end{center}
\end{frame}


\begin{frame}{Autonomous driving}

The agent (the car) is on a two-way road with a car in front of it,
\begin{itemize}
	\item it can {\green stay behind} {\green (safe/slow)};
	\item it can {\red overtake} {\red (unsafe/fast)}.
	\end{itemize}

\begin{center}
\includegraphics[trim={0 0cm 0 0.8cm}, clip, width=0.5\textwidth]{../../source/img/highway.pdf}\\
\href{https://budgeted-rl.github.io/\#driving-styles}{\includegraphics[width=0.5\linewidth]{img/highway_env}}
\end{center}
\end{frame}

\begin{frame}{Risk-sensitive exploration}
\begin{alertblock}{How to collect the batch $\cD$?}
We propose an $\epsilon$-greedy exploration procedure:
\pause
\begin{itemize}[<+->]
	\item Sample an initial budget $\beta_0\sim\cU(\cB)$
	\item At each step, where $\os=(s,\beta)$ only explore feasible budgets:
	\begin{align*}
	&\oa = (a, \beta_a)\sim\mathcal{U}(\Delta_{\cA\cB})\\
	&\text{ where }  \Delta \text{ is such that }\probability{a, \beta_a|s, \beta} \text{verifies} \expectedvalue[\beta_a]\leq\beta
	\end{align*}
\end{itemize}
\end{alertblock}
\end{frame}

\begin{frame}{Corridors}

Two corridors:
\begin{enumerate}
\item one with {\red high costs / high rewards}
\item the other with {\green no costs / low rewards}
\end{enumerate}

\begin{center}
	\includegraphics[width=0.3\textwidth]{../../source/img/corridors_paths}
\end{center}

\begin{itemize}
	\item[$\rightarrow$] Validate the \alert{risk-sensitive exploration} procedure
\end{itemize}

\end{frame}


\begin{frame}{Corridors}

\begin{columns}
\begin{column}{0.5\linewidth}
\includegraphics[trim={0.5cm 0.1cm 0.7cm 0.2cm}, clip, page=1, width=\linewidth]{../../source/img/corridors}
\end{column}
\begin{column}{0.46\linewidth}
\begin{center}
\href{https://budgeted-rl.github.io/\#risk-sensitive-exploration}{\includegraphics[trim={2cm 0.1cm 1cm 0.5cm}, clip, width=\linewidth]{../../source/img/corridors_densities.pdf}}
\end{center}
\end{column}
\end{columns}
\end{frame}


\end{document}
